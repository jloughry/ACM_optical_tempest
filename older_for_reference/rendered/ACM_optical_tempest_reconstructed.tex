Information Leakage from Optical Emanations

A previously unknown form of compromising emanations has been discovered. LED status indicators on data communication equipment, under certain conditions, are shown to carry a modulated optical signal that is significantly correlated with information being processed by the device. Physical access is not required; the attacker gains access to all data going through the device, including plaintext in the case of data encryption systems. Experiments show that it is possible to intercept data under realistic conditions at a considerable distance. Many different sorts of devices, including modems and Internet Protocol routers, were found to be vulnerable. A taxonomy of compromising optical emanations is developed, and design changes are described that will successfully block this kind of ``Optical TEMPEST'' attack.

Much of this work was done while J. Loughry was a graduate student in the Department of Com- puter Science and Software Engineering at Seattle University.
Authors’ addresses: J. Loughry, Lockheed Martin Space Systems, Dept. 3740, Mail Stop X3741, P.O. Box 179, Denver, CO 80201; email: loughry@lmco.com; D. A. Umphress, Auburn University, Department of Computer Science and Software Engineering, 215 Dunstan Hall, Auburn University, AL 36849; email: umphress@eng.auburn.edu.

Introduction

Can optical radiation emitted from computer LED (light emitting diode) status indicators compromise information security? Data communication equipment, and even data encryption devices, sometimes emit modulated optical signals that carry enough information for an eavesdropper to reproduce the entire data stream being processed by a device. It requires little apparatus, can be done at a considerable distance, and is completely undetectable. In effect, LED indicators act as little free-space optical data transmitters, like fiber optics but without the fiber.

Fig. 1. Compromising optical emanations. The lower trace shows the ±15 V EIA/TIA-232-E input signal (at 9600 bits/s); the upper trace shows optical emanations intercepted 5 m from the device.

Experiments conducted on a wide variety of devices show evidence of ex- ploitable compromising emanations in 36% of devices tested. With inexpensive apparatus, we show it is possible to intercept and read data under realistic conditions from at least across the street. In Figure 1, the lower trace shows the ±15 V EIA/TIA-232-E waveform of a serial data signal at 9600 bits/s. The upper trace shows modulated optical radiation intercepted 5 m from the device. A high correlation is evident.

We have successfully recovered error-free data at speeds up to 56 kbits/s; the physical principles involved ought to continue to work up to about 10 Mbits/s. Protecting against the threat is relatively straightforward, but may require design changes to vulnerable equipment.

Organization

The first part of this article reviews the idea of compromising emanations, and gives an overview of what information is to be found in the literature. Next comes a technical explanation of why compromising optical emanations exist, together with some of their properties. A series of experiments is then described, along with results that were found. Finally, some possible countermeasures are discussed, along with directions for future work. Related work on active attacks using optical emanations is presented in the appendices.

EMSEC, TEMPEST, and Compromising Emanations

Compromising Emanations [National Computer Security Center 
1988]: ìUnintentional data-related or intelligence-bearing signals 
that, if intercepted and analyzed, disclose the information 
transmi[tted], received, handled, or otherwise processed by any information 
processing equipment. See TEMPEST.î 

Thorough discussion of compromising emanations and EMSEC (emissions 
security) in the open literature is limited. The information that is available 
tends to exhibit a strong bias toward radio frequency (RF) emanations from 
computers and video displays. Because of the high cost of equipment and the 
difficulty of intercepting and exploiting RF emanations, reports of successful 
attacks against emanations have been limited primarily to high-value sources 
of information such as military targets and cryptologic systems. A significant 
problem is that much important information on compromising emanations is 
classified [Russell and Gangemi 1991], although some documents have recently 
been declassified [National Security Agency 1992, 1995, 1994]. 
2.1 Related Work 
The ability to compromise signals emanating from computers has been known 
for some time. For instance, Smulders [1990] found RF emanations in unshielded 
or poorly shielded serial cables, and van Eck [1985] showed that 
cathode-ray tube video displays can be read at a distance by intercepting and 
analyzing their RF emanations. Others have noted RF compromise, including 
more contemporary research showing ways to hide information in signals emitted 
by video devices as well as specialized fonts that minimize compromising RF 
emanations [Kuhn and Anderson 1998]. Wright [1987] described, anecdotally, 
the discovery of electrically conducted compromising emanations from cipher 
machines as early as 1960. For an excellent overview of the current state of 
emanations security research, the interested reader is referred to the book by 
Anderson [2001] and a related paper by Kuhn and Anderson [1998]. 
Until recently, little mention of signals in the optical spectrum was found in 
the literature. Kuhn [2002] has demonstrated remotely reading CRT displays 
from diffuse optical emanations, without requiring line-of-sight access. Other 
related topics include security of fiber optics [Hodara 1991; EXFO Electro- 
Optical Engineering, Inc. 1999] and optical communications [Wilkins 1641]. 
Social engineering attacks such as ìshoulder surfingî and visual surveillance 
of video displays are well covered in Fites and Kratz [1993]. Free-space optical 
data links are prone to interception, and for this reason wireless data 
links (both laser and RF) are typically encrypted [Lathrop 1992]. But with the 
exception of a work of fiction, in which one character uses the LEDs on a computer 
keyboard1 to send information in Morse code [Stephenson 1999], and 
inferences from redacted sections of partially declassified documents [National 
Security Agency 1992], a thorough search of the literature revealed no direct 
1See also Appendix A. 
ACM Transactions on Information and System Security, Vol. 5, No. 3, August 2002.
Information Leakage from Optical Emanations ≤ 265 
mention of the risk of interception of data from optical emanations ofLEDstatus 
indicators. 
3. COMPROMISING OPTICAL EMANATIONS 
ìThe [IBM] 360 had walls of lights; in fact, the Model 75 had so many 
that the early serial number machines would blow the console power 
supply if the ëLamp Testí button was pressed.î [Morris 1996] 
3.1 Light-Emitting Diodes 
Light-emitting diodes are cheap, reliable, bright, and ubiquitous. They are used 
in nearly every kind of electronics, anywhere a bright, easy-to-see indicator is 
needed. They are especially common in data communication equipment. Every 
year, some 20ñ30 billion LEDs are sold [Perry 1995]. 
LEDs are very fast; that is, they exhibit a quick response to changes in the 
applied drive voltage (tens of nanoseconds). In fact, common visible LEDs are 
fast enough that a close cousin is used as a transmitter in fiber optic data links 
at speeds in excess of 100 Mbits/s [HewlettñPackard Company 1993b]. 
Although fast response time is oftentimes a desirable quality in a display, 
LEDs are fast enough to follow the individual bit transitions of a serial data 
transmission. Herein lies the problem: if certain LED indicators are visible to 
an attacker, even from a long distance away, it becomes possible for that person 
to read all of the data going through the device. 
One of the advantages of LED displays is that they can be read from across 
a room. The disadvantage may be that they can be read from across the street. 
3.2 Rationale for the Existence of Compromising Optical Emanations 
The brightness of LED displays would not be a problem if it were not for the way 
they interact with serial data transmissions. Consider the idealized EIA/TIA- 
232-E waveform and associated LED response curve depicted in Figure 2. The 
upper waveform shows the EIA/TIA-232-E serial data signal; the lower waveform 
illustrates the optical output of an LED indicator monitoring that signal. 
As long as the rise time of the LED is less than 12 
of the unit interval tUI, the 
LED will accurately enough mirror the EIA/TIA-232-E data signal at the critical 
points shown by the small circles in the diagram to enable recovery of the 
original data. 
The EIA/TIA-232-E standard (formerly known as RS-232) defines a bit-serial 
format using bipolar encoding and non-return-to-zeroñlevel (NRZñL) signaling 
[Electronic Industries Association, Engineering Department 1991]. As illustrated 
in Figure 3, bits are transmitted asynchronously, with framing bits 
embedded in the serial data stream for synchronization between sender and 
receiver. During periods when no data are being transmitted, the transmitter 
remains in the logical ì1î state. The start of a new symbol is indicated by a momentary 
excursion to the logical ì0î state for one unit interval, called the start 
bit. This is followed by a serial waveform consisting of a mutually agreed-upon 
ACM Transactions on Information and System Security, Vol. 5, No. 3, August 2002.
266 ≤ J. Loughry and D. A. Umphress 
Fig. 2. EIA/TIA-232-E serial data waveform and typical LED response. 
Fig. 3. EIA/TIA-232-E serial data waveform and maximum jitter tolerance from TIA/EIA-404-B. 
number of data bits, sent least significant bit first. Following the last data bit, 
the transmitter returns to the logical ì1î state for at least one unit interval, 
called the stop bit, in order to provide necessary contrast for the receiver to 
recognize the beginning of the next start bit. (Another way of looking at this 
is that the channel is required to return to the idle state for at least one unit 
interval between characters.) 
EIA/TIA-232-E uses bipolar encoding, with a negative voltage signifying 
logical ì1î and a positive voltage used for logical ì0î [Black 1996]. Usually, 
LEDs are wired to light up for a logical ì0î so that they flicker when bits are 
ACM Transactions on Information and System Security, Vol. 5, No. 3, August 2002.
Information Leakage from Optical Emanations ≤ 267 
transmitted, and remain dark when the channel is idle. The fact that the original 
signal is bipolar is immaterial. As long as theLEDis fast enough to faithfully 
reproduce the timing of bit transitions, the optical output will contain all of the 
information in the original EIA/TIA-232-E signal. 
LEDs cannot be connected directly to logic circuits, as they would draw too 
much power from the signal source. For reasons of cost, however, the very same 
high-speed gates (usually TTL or CMOS inverters) typically used to construct 
logic circuits are also employed to power the LEDs [Lancaster 1980]. The result 
is a direct path allowing information to flow from the serial data channel to the 
optical output of the LED. Because the monitoring circuit was not designed for 
the purpose, the resulting optical signal may exhibit noise or other degradation, 
but LEDs and their associated driver circuitry are generally more than fast 
enough to reproduce a serial data signal at normal data rates. 
3.2.1 Characteristics of the Optical Signal. NRZñL signals are susceptible 
to noise, which is why other signaling methods, such as differential Manchester 
encoding, are most often used in long-distance digital communication systems. 
To overcome the noise sensitivity of NRZñL, additional redundancy is often 
introduced into the communication channel in the form of channel encoding 
[Proakis and Salehi 1994]. Parity checks, cyclic redundancy checking (CRC), 
and other error detection and correction methods may be used to increase the 
reliability of the system. But it should be noted that these features are also 
available to an eavesdropper, who may use them to overcome the effects of a 
poor optical signal. 
As optical communication systems go, it must be recognized that LED status 
displays are highly suboptimal. There are no beam-forming optics on the transmitting 
LED. The radiant flux available is extremely limited. Buffer circuits 
used to drive LED indicators, while more than fast enough for their intended 
purpose, are not optimized for high-speed data transmission in the way that 
special-purpose circuits used in fiber optic transmitters are. Practical optical 
data communication systems use laser transmitters, sophisticated encoding 
schemes, and coherent detectors that greatly improve signal recovery under 
noisy conditions [Gagliardi 1995]. Our hypothetical eavesdropper would likely 
have to deal with off-axis aiming errors, high levels of optical background noise 
from artificial lighting, and lack of a priori knowledge of the specific bit rate 
and word length used by the target. Nevertheless, our experiments show that 
with a sensitive detector and telescopic optics, it is possible for an eavesdropper 
to recover a noisy analog waveform closely approximating the original digital 
data stream. Once the received optical signal has been amplified, cleaned of 
noise, and fed to a USART (Universal Synchronous-Asynchronous Receiver- 
Transmitter)óan inexpensive chip that serves as a ready-made solution to the 
problem of decoding a noisy signalóthe original data stream is easily recovered. 
3.2.2 Insensitivity to the Modulation Scheme Employed. High-speed 
modems employ a variety of complicated modulation schemes, including frequency, 
amplitude, and phase modulation to maximize available bandwidth on 
voice-grade telephone lines. But this makes no differenceóit is the relatively 
ACM Transactions on Information and System Security, Vol. 5, No. 3, August 2002.
268 ≤ J. Loughry and D. A. Umphress 
Table I. Proposed Classification System for Optical Emanations 
Type Correlated to Associated Risk Level 
Class I State of the device Low 
Class II Activity level of the device Medium 
Class III Content (data) High 
simple NRZñL waveform of the EIA/TIA-232-E data signal that is modulated 
onto the LED. 
3.2.3 Nonsusceptibility of Other Light Sources. Questions remain as to the 
susceptibility of non-LED sources to interception of compromising optical emanations. 
Liquid crystal (LCD) displays, in particular, exhibit a relatively slow 
impulse response, typically on the order of tens of milliseconds, making these 
displays relatively poor sources of compromising optical emanations, except at 
fairly low data rates. Cathode ray tube (CRT) displays, however, at the pixel 
level, are very fast, and have been shown to be vulnerable even to nonñline-ofsight 
optical emanations [Kuhn 2002]. 
3.3 Classification of Optical Emanations 
It is useful to consider a division of optical emanations into three broad classes 
according to the amount of information potentially carried to an adversary. The 
proposed taxonomy is shown in Table I. In the list that follows, LED indicators 
that exhibit Class n behavior are called Class n indicators. 
The classifications are: 
óClass I indicators, which are unmodulated. The optical emanations put out 
by this type of display are constant, and correlated with the state of a device 
or communication channel. Class I indicators communicate at most one bit 
of information to an observer. An example would be a power-on indicator. 
óClass II indicators are time-modulated, and correlated with the activity level 
of a device or communication channel. Class II indicators provide an adversary 
with considerably more information than Class I indicators do. On face 
value, while the content of the data being processed by a device is not known, 
the fact that something is being transmitted, and a rough idea of where and 
how much, together make possible traffic analysis of interesting targets. Examples 
of Class II indicators include the Work Station Active light on an 
IBM 5394 Control Unit, activity indicators on Ethernet interfaces, and the 
front-panel lights of a Cisco router. It is important to note that by affecting 
the activity level of a device, and hence modulating the output of a Class II 
indicator, it is possible for an attacker to implement a covert timing channel. 
óClass III optical emanations are modulated optical signals that are strongly 
correlated with the content of data being transmitted or received. If the correlation 
is sufficiently good, then from analysis of Class III optical emanations 
it is possible to recover the original data stream. Examples of Class III emanations 
are surprisingly common; the ìTransmitted Dataî and ìReceived 
Dataî indicators on modems are usually Class III. 
ACM Transactions on Information and System Security, Vol. 5, No. 3, August 2002.
Information Leakage from Optical Emanations ≤ 269 
Devices having at least one Class II indicator, but no Class III indicators, are 
called Class II devices; any device having at least one Class III indicator is a 
Class III device. Class III devices are the most interesting. 
Note that in both the Class I and Class II cases, the adversary gets no more 
information than the operator does; the indicator is being used in the manner 
for which it was intended, except that the eavesdropper is unauthorized, and 
reading the information at a distance. 
Class III devices may arise when the designer of a device inadvertently specified 
a Class III indicator where a Class II indicator was needed. It is not clear 
whether there is any situation in which a Class III indicator would be warranted, 
except in the case of an extremely low-speed communication channel, 
where individual bit transitions could be observed by eye and decoded. In most 
cases, the activity of a data communication channel occurs too fast for the human 
eye to follow. In the real world, an oscilloscope is a much more useful tool 
than a Class III indicator. 
Potentially dangerous Class III indicators can be converted to the safer and 
more useful Class II type by the addition of a pulse stretching circuit, as described 
in Section 6 on Countermeasures below. 
4. EAVESDROPPING EXPERIMENTS 
Three series of experiments were run. First, a survey was made of a large 
number of devices, looking for evidence of Class III behavior. Then, long-range 
testing was done on a selection of devices, to prove the feasibility of interception 
under realistic conditions. Finally, examination was made of the internals 
of several devices, in an attempt to understand why these emanations 
occur. 
4.1 Hypothesis 
The null hypothesis was stated as follows: ìIt is not possible to recover data 
from optical emanations.î The null hypothesis was disproved by experiment. 
4.2 Experimental Design and Methodology 
A total of 39 devices containing 164 unique LED indicators were identified for 
this study. The devices selected for testing were chosen to represent a wide 
variety of information processing technology, including low-speed and highspeed 
communication devices, local-area network (LAN) and wide-area network 
(WAN) devices, PC and mainframe computers, mass storage devices, and 
peripherals. 
Prior to commencement of measurements, radiometric readings were taken 
on an optical bench of a standard red LED driven by a squarewave signal. These 
measurements were used to establish a baseline. Following this step, each of 
the 164 LED indicators identified in the survey was examined for evidence of 
Class III behavior. 
Measurements were made of individual LED indicators by placing a hooded 
detector in contact with each LED. A dual-trace oscilloscopewas used to observe 
ACM Transactions on Information and System Security, Vol. 5, No. 3, August 2002.
270 ≤ J. Loughry and D. A. Umphress 
the signal from the detector. To visualize the corresponding data stream, a 
breakout box was inserted into the data path, with the original data displayed 
alongside the optical signal from the detector. 
The detector used was a high-speed, large-area silicon PIN (Positiveñ 
IntrinsicñNegative) photodiode with an active area of 1 mm2. The responsivity 
of this detector is 0.45 A/Wat a nominal wavelength of 830 nm, with a spectral 
response of 350ñ1100 nm. The photocurrent from the detector was amplified 
by a transimpedance photodiode amplifier operated in zero-bias mode. Signals 
were observed with a 200-MHz digital oscilloscope, and captured for later 
analysis. 
The bandwidth of the photodiode amplifier is inversely proportional to its 
gain setting; at a gain factor of 107 V/A, the bandwidth of the detectorñamplifier 
system is only 10 KHz. Therefore, for most measurements, the amplifier was 
operated at a gain setting of 104 V/A, yielding an overall detectorñamplifier system 
bandwidth of 45 KHz, which was marginal, but adequate. For higher-speed 
measurements, the photodiode was connected directly to the input amplifier of 
the oscilloscope and operated in the quadrant IV (photovoltaic) region. Limited 
sensitivity in this configuration is what necessitated placing the detector directly 
in contact with the LED. However, the greatly increased bandwidth of 
the detectorñamplifier system in this configuration allowed for examination of 
very high speed devices for evidence of signals in the MHz range. 
4.2.1 Long-Range Testing. Long-range optical eavesdropping experiments 
were conducted with a small number of representative devices. The ANP Model 
100 short-haul modem, Hayes Smartmodem OPTIMA 9600 and 14400, and a 
Practical Peripherals PM14400FXMT fax modem were all examined. 
The same photodetector and amplifier system described in the previous section 
was used. The detector was mounted at the focus of an optical system 
consisting of a 100 mm diameter, f =2:5 converging lens, an aperture stop, and 
a 650-nm optical bandpass filter, chosen to match the spectral output of a standard 
visible red LED [Agilent Technologies 1999]. 
The device under test was placed a measured distance away, and connected 
to an identical unit at the test station through a length of unshielded twisted 
pair cable. The image from a single LED on the device under test was adjusted 
to completely cover the detectorís active area. Test transmissions were made 
to each device, and the EIA/TIA-232-E waveform and resulting optical signals 
captured for analysis. 
4.2.2 Experimental Methodology. Three independent variables and one 
dependent variable were identified. The independent variables were: (1) the 
separation distance between the detector and the device under test, (2) the data 
transmission rate, and (3) ambient lighting conditions on the test range. The 
dependent variable was the correlation between the received optical signal and 
the original EIA/TIA-232-E waveform captured at the same time. The independent 
variables were varied according to a formal test matrix. Separation 
distance was varied from 5 m to 38 m (the maximum dimension of the laboratory) 
in increments of 5 m during the test. At each measured distance, test 
ACM Transactions on Information and System Security, Vol. 5, No. 3, August 2002.
Information Leakage from Optical Emanations ≤ 271 
transmissions were made at data rates of 300, 600, 1200, 2400, 4800, 9600, and 
19 200 bits/s. 
For simplicity, symbols in the optical signal were detected by observing the 
signalís amplitude at one-half of the unit interval after the NRZñL transition. 
Because this was a proof-of-concept experiment, actual bit-error rates were not 
measured. The optical waveform from the detector amplifier was compared to 
the original EIA/TIA-232-E signal waveform obtained from a breakout box inserted 
in the data path between the data generator and the device under test. 
After each series of measurements over the full range of distances, the ambient 
lighting conditions on the test range were changed. Lighting conditions 
tested included daylight office conditions (i.e., sunlight coming through windows, 
plus artificial light), normal fluorescent office lighting, nighttime office 
lighting (scattered fluorescent lights plus some light entering through windows 
from the streetlights outside), and a darkened, windowless conference room. An 
optical bandpass filter was used in some tests in an attempt to reduce the level 
of background radiation and determine if detector overload was an important 
factor. All tests were conducted indoors. 
4.3 Experimental Results 
Results of the survey of devices are shown in Table II. Of 39 devices tested, 14 
showed evidence of Class III optical emanations at the tested bit rate. 
4.3.1 Results of the Survey of Devices. Dial-up and leased-line modems 
were found to faithfully broadcast data transmitted and received by the device. 
Only one device of this type did not exhibit Class III emanations: the Practical 
Peripherals PM14400FXMT fax modem. The shortest pulse duration measured 
from this device was 20 ms, even at high data rates. 
None of the LAN interface cards tested, including 10 Mbits/s Ethernet and 
16-Mbits/s Token Ring adapters, were found to broadcast any recognizable data. 
Examination of the data sheet for a chipset used in fiber-optic Ethernet devices 
reveals a possible reason for this finding. According to the HewlettñPackard 
Company [1993a], LED drivers for transmit, receive, and collision indicators 
are filtered through pulse stretching circuits to make their activity more 
visible. The pulse stretcher extends the on-time of LED indicators to a minimumof 
several milliseconds. This makes short pulses easier to see, but severely 
limits the bandwidth of the LED from the perspective of compromising optical 
emanations. All of the Ethernet and Token Ring devices examined showed similar 
behavior in this regard. 
Both of the routers tested (Cisco Series 4000 and 7000 routers equipped with 
Token Ring, Fast Serial and FDDI Interface Processors) were found to broadcast 
Class III emanations from the Fast Serial LEDs on their back panels. Frontpanel 
activity indicators, while suggestive of data leakage, typically exhibited 
a typical minimum pulse width on the order of 20 ms, indicating that the frontpanel 
indicators are merely Class II. None of the LAN devices tested showed 
any evidence of Class III emanations from LAN traffic. 
Two T1 (1.554 Mbits/s) CSU/DSU (Channel Service Unit/Data Service Unit) 
devices were tested. Neither unit showed evidence of Class III emanations. 
ACM Transactions on Information and System Security, Vol. 5, No. 3, August 2002.
272 ≤ J. Loughry and D. A. Umphress 
Table II. Results of a Survey of 39 Devices 
LED Indicator Class I Class II Class III 
Modems and Modem-Like Devices 
ANP Model 100 short-haul modem, TD indicator ≤ ANP SDLC card, TD indicator ≤ CASE/Datatel DCP3080 CSU/DSU, TD indicator ≤ Hayes Smartmodem OPTIMA 14400, SD indicator ≤ Hayes Smartmodem OPTIMA 9600, SD indicator ≤ Motorola Codex 6740 Hex TP card, TD indicator ≤ Motorola Codex 6740 TP Proc card, TD indicator ≤ MultiTech MultiModem V32, TD indicator ≤ Practical Peripherals PM14400FXMT fax modem, TX and 
RX indicators ≤ SimpLAN IS433-S printer sharing device, front panel LEDs ≤ Telemet SDR-1000 Satellite Data Receiver, Data indicator ≤ V.32bis modem simulator, TD indicator ≤ 
LAN Devices 
3Com TokenLink III Token Ring LAN card, Link indicator ≤ Cabletron TRXI-24A Token Ring hub, Activity indicator ≤ Ethernet NIC, unknown manufacturer, backplane LED ≤ Ethernet AUI, unknown manufacturer, Link indicator ≤ Ethernet AUI, unknown manufacturer, Receive indicator ≤ Ethernet AUI, unknown manufacturer, Transmit indicator ≤ Synoptics 2715B Token Ring hub, Link indicator ≤ 
WAN Devices 
Cisco 4000 IP router, Fast Serial TD indicator ≤ Cisco 4000 IP router, front panel LED ≤ Cisco 7000 IP router, Fast Serial TD indicator ≤ Cisco 7000 IP router, front panel LED ≤ Stratacom IPX SDP5080A, RXD indicator ≤ Verilink FT1 DSU/CSU, Pulses indicator ≤ Westel 3110-30 DS1 Connector, Pulses indicator ≤ 
Storage Devices 
2x CD-ROM drive, unknown manufacturer, activity LED ≤ Compaq Proliant hot-swappable disk array, activity LED ≤ Compaq Proliant server, floppy drive LED ≤ 
IBM 4702 controller, 5 14 
-inch floppy drive LED ≤ IBM 4702 controller, hard disk activity LED ≤ IBM 8580 computer, disk activity indicator ≤ PC, unknown manufacturer, hard disk LED ≤ 
Miscellaneous Devices 
Hewlett-Packard LaserJet 4 laser printer, Ready indicator ≤ IBM 3745 Front-End Processor, console LEDs ≤ IBM 4019 Laser Printer, Buffer indicator ≤ IBM 5394-01B Control Unit, Work Station Active LED ≤ IBM AS/400 Model 9406, Processor Activity indicator ≤ WTI POLLCAT III PBX Data Recorder, PBX Input A, B 
indicators ≤ 
ACM Transactions on Information and System Security, Vol. 5, No. 3, August 2002.
Information Leakage from Optical Emanations ≤ 273 
Fig. 4. Degradation of the optical signal with increasing distance from the target. The data rate 
was 9600 bits/s. 
Lower-speed CSU/DSU devices, however, on 56-kbits/s leased circuits, behaved 
similarly to dialup modems. All showed usable Class III emanations in both 
synchronous and asynchronous operation. 
Intelligent serial data switches (i.e., printer sharing devices), a satellite data 
receiver, and a PBX call data recorder behaved similarly to the modems in this 
test. Data from attached devices showed up in the form of Class III optical 
emanations on the front panels of all these devices. 
Mass storage devices such as hard disks and tape transports are usually 
equipped with device activity indicators. It was hypothesized that the optical 
output of these LEDs might be related to data transfers to or from the storage 
device. A variety of PC and minicomputer hard disk drives, floppy diskette 
drives, CD-ROM drives and tape transports were tested. None were found to 
emit anything other than Class II optical emanations. 
Miscellaneous devices tested included the Processor Activity indicator 
on an IBM AS/400 computer, the Work Station Active indicator on an IBM 
5394 terminal controller, and control panel indicators on IBM and Hewlett- 
Packard laser printers. All of these devices were found to be Class II at 
most. 
No significant difference was found between the observability of 5-mm 
standard-sized LEDs and the much smaller surface-mount components 
used in newer devices. The absolute brightness levels of these LEDs are 
comparable. 
4.3.2 Long-Range Testing. Results of long-range testing are shown in 
Figure 4. Note the increasing signal degradation as the distance was varied 
from 10 m to 30 m from the detector. There is a high correlation evident between 
the EIA/TIA-232-E waveform and the received optical signal, as shown 
in Figure 5. For comparison, the correlation between the upper trace of the first 
part of Figure 4 and a random signal of similar amplitude to the optical signal 
was found to be ∞0:02558, which is not statistically significant. 
No difference was seen at faster bit rates. Interestingly, several devices continued 
to emit a recognizable optical signal at data rates exceeding the rated 
ACM Transactions on Information and System Security, Vol. 5, No. 3, August 2002.
274 ≤ J. Loughry and D. A. Umphress 
Fig. 5. Observed correlation k between the original EIA / TIA-232-E data signal (9600 bits/s) and 
the received optical signal for distances of 5 m, 10 m, 20 m, and 30 m. This is from the data of 
Figures 1 and 4. 
capability of the device. Despite high noise levels in the recorded waveforms, 
due apparently to a combination of detector shot noise and thermal noise in the 
amplifier, signals were intercepted and properly decoded at a distance. 
4.3.3 Reverse Engineering of Devices. It appears that some types of data 
encryption devices, in particular stand-alone data encryptors and modems with 
built-in link encryption capability, may emit optical signals in unencrypted 
form. 
Figure 6 is a detail taken from the Installation and Operation Manual for 
the Paradyne InfoLock model 2811-11 DES encryptor. The InfoLock 2811 is a 
standalone DES (Data Encryption Standard) link encryptor of the type used 
by financial institutions to encrypt data on their wire transfer and ATM (automated 
teller machine) networks [Paradyne Corporation 1985]. 
The figure shows a portion of the data path between the DTE connector (Data 
Terminal Equipmentóthe side of the encryptor that connects to a computer) 
through the encryption function, to the DCE connector (Data Communications 
Equipmentóthe side that connects to a modem). The DTE, or red side, is unencrypted; 
the DCE, or black side, is encrypted [United States Department of 
Defense 1987]. It is clear from this diagram that LED indicators on the TXD 
and RXD (transmitted and received data, respectively) are on the red side of 
the InfoLock 2811. This is a serious design flaw. The LEDs will display all of 
the data passing through the device (in either direction) in its unencrypted, or 
plaintext, form. 
It is believed that any link encryption device with LED indicators may potentially 
contain this flaw. Modems with built-in link encryption are probably 
vulnerable as well. Stand-alone data encryptors like the InfoLock 2811 will 
protect downstream equipment on the black side, but are vulnerable to compromising 
optical emanations themselves. The failure mode results in leakage 
ACM Transactions on Information and System Security, Vol. 5, No. 3, August 2002.
Information Leakage from Optical Emanations ≤ 275 
Fig. 6. Detail from Installation and Operation Manual for the InfoLock 2811-11 DES encryptor. 
of cleartext. The determination of whether or not a particular encryption unit 
is vulnerable will require examination of the internals of each device. 
5. INTERPRETATION OF RESULTS 
The null hypothesis was disproved. 
Class III emanations were found only in data communication devices, but 
not all data communication devices examined were found to be Class III. In 
particular, none of the LAN cards tested were found to exhibit Class III behavior 
(although most of them were Class II). No data storage device was found to be 
Class III. The design flaw in the InfoLock 2811 encryption device is particularly 
interesting. 
Optical background noise from artificial sources proved to be a significant 
problem. Sources of low-frequency noise (120 Hz and below) include incandescent, 
fluorescent, mercury vapor and sodium vapor lamps; high-frequency noise 
sources include industrial fluorescent lighting and compact fluorescent lights. 
Sunlight, while dwarfing in brightness the artificial sources, contributes only 
a DC component, and is much easier to filter out. Artificial lighting sources 
proved to be a pervasive and difficult-to-eliminate source of problems, because 
they contribute large amounts of amplitude-modulated noise containing strong 
ACM Transactions on Information and System Security, Vol. 5, No. 3, August 2002.
276 ≤ J. Loughry and D. A. Umphress 
harmonics in precisely the range of interest. In the United States, the standard 
60-Hz alternating current electrical power supply leads to a characteristic noise 
component at 120 Hz. Most common data transmission rates are multiples of 
this frequency, complicating recovery of the data. 
Digital signal processing techniques can help. By using a low-pass filter to 
isolate the 120-Hz component of the received optical signal, low-frequency noise 
can be isolated and subtracted from the optical signal, yielding a new signal 
without the 120-Hz component. Results of experiments in this area were very 
encouraging. Experiments using analog electronic filters were also encouraging. 
The limiting factors in long-range interception seem to be the optics and 
the detectorñamplifier system. Both a larger aperture and a narrower field of 
view are required. It is believed that, out to a range of at least several hundred 
meters, the optical flux available from a single LED is well within the capability 
of our detector. The response time of a typical LED suggests a practical upper 
limit on the order of 10 Mbits/s. Clearly, however, interception of data at longer 
ranges and higher speeds is feasible. 
6. COUNTERMEASURES 
A contributing factor to the threat of optical interception is a historical tendency 
to locate computers and data communication equipment in environmentally 
controlled ìglass housesî which provide no barrier to the escape of optical 
radiation. Clearly, this must now be considered a threat. 
Examination of lighted windows of high-rise office buildings in the evening 
hours reveals a rich variety of equipment racks with LED indicators in view. 
Line-of-sight access is surprisingly easy to find. Fortunately, optical emanations 
are easier to contain than RF; opaque materials will shield the radiation 
effectively. 
Black tape over the LEDs is effective, but inelegant. The best solution to 
the problem is a design change. Status displays could be designed to be deactivated 
when not in use (effectively making them Class I), or alternative display 
technologies could be employed, such as LCD and displays, which can be made 
inherently Class II due to their relatively slow impulse response. But many of 
these other technologies (such as CRT displays) are more expensive. LEDs are 
fast, cheap, and relatively low power indicators that can be read from across 
a room (a significant weakness of liquid crystal displays). It is preferable to 
retain these desirable properties. 
A better solution is presented in Figure 7. The key here is a violation of 
the worst-case jitter tolerance of the serial data communication transmission 
scheme in use [Telecommunications Industry Association 1996]. If the minimum 
on-time of an LED indicator is greater than 1.5 times the unit interval 
of the current data rate,2 then an attacker will be unable to recover sufficient 
information to decode the signal. The effect is to convert a Class III indicator to 
Class II. The resulting low-pass filter removes a sufficient amount of information 
from the optical signal that an attacker cannot recover the original data 
from the emanations. The LED will flicker in response to a random data signal, 
2Or alternatively, the slowest data rate expected. 
ACM Transactions on Information and System Security, Vol. 5, No. 3, August 2002.
Information Leakage from Optical Emanations ≤ 277 
Fig. 7. Effect of a pulse stretcher between the data source and LED. Vertical lines are decision 
points; the small circle indicates the point at which an eavesdropper would incorrectly read the 
optical signal. 
and hence will still be useful as a Class II activity indicator, but the risk of 
significant information leakage is reduced. 
More conservatively, the minimum on-time of the LED could be made to be at 
least twice the unit interval; even more conservatively, the minimum off-time 
could be similarly controlled as well. Most conservatively of all, the minimum 
on-time of the LED should be made to equal the nominal character interval of 
the current data rate, or of the slowest data rate expected. This will guarantee 
that an attacker cannot derive any information from the optical signal other 
than that a symbol was transmitted. 
Even though it appears that at least one device (the PM14400FXMT fax 
modem) already incorporates pulse stretching functionality on its status LEDs, 
it is believed that this was done to make the display easier to read, not for 
reasons of blocking compromising emanations [Johnson 1995]. 
Of course, even in the presence of the aforementioned hardware modification, 
a patient attacker might simply time-modulate the asynchronous data stream 
in such a way as to effect a covert channel at a rate of ( tUI 
tcharacter 
)∞1 bits/s. It is 
difficult to completely eliminate the possibility of covert timing channels in 
multilevel systems [Proctor and Neumann 1992]. 
7. SUMMARY AND CONCLUSIONS 
Modulated optical radiation from LED status indicators appears to be a previously 
unrecognized source of compromising emanations. This vulnerability is 
exploitable at a considerable distance. Primarily, data communication equipment 
is affected, although data encryption devices also pose a high risk of 
information leakage, potentially leading to loss of plaintext and encryption 
keys. 
ACM Transactions on Information and System Security, Vol. 5, No. 3, August 2002.
278 ≤ J. Loughry and D. A. Umphress 
A taxonomy of optical emanations was developed according to the amount of 
useful information available to an attacker. Experiments showed that Class III 
optical emanations, which should never be permitted, were present in 36% of 
devices tested, and data could be read from these devices at a distance of at least 
20 m. Countermeasures are possible that will convert a vulnerable Class III 
indicator into the safer (but still useful) Class II variety, by means of inserting 
a pulse stretcher into the LED driver circuitry. 
7.1 Conclusions 
Theft of information by interception of optical emanations is necessarily limited 
to one-wayóthe intruder can only receive information. However, login IDs 
and reusable passwords obtained in this fashion could be used to support a conventional 
attack. As mentioned before, parity checking, CRC values, and other 
error detection and correction features embedded in the data stream are available 
to the eavesdropper too, and can be of great benefit in helping to overcome 
the effects of a low-quality optical signal. 
Ironically, it may be the simplest devicesólow-speed, obsolete, insignificant 
parts of a networkóthat provide a gateway for intruders. In our experiments, 
it was low-speed modems, routers, line drivers, data loggers, and a printer 
sharing device that were found to be the most enthusiastic broadcasters of 
data. Class III optical emanations have been observed in the wild from devices 
as diverse as TTY-equipped payphones in airports and the digital control box 
of a player piano. Like the Purloined Letter, they hide in plain sight: a tangle 
of remote office connections in the corner, a modem sitting next to a PC by the 
window, or a call-accounting system on the PBX. 
7.2 Summary of Contributions 
óThe existence of compromising optical emanations was proved. 
óSuccessful exploitation, under realistic conditions, was demonstrated. 
óA taxonomy of compromising optical emanations was developed. 
óSome possible countermeasures were presented. 
8. DIRECTIONS FOR FUTURE RESEARCH 
Much work remains to be done. While we have shown that it is possible to 
intercept data at realistic data rates out to a few tens of meters, the maximum 
distance at which this can be accomplished remains unknown. Improved 
signal detection techniques, optics, and detectors would go a long way toward 
quantifying the effective limits on distance and bit error rate. 
Other possible areas of investigation include the exploitation of Class II devices 
(especially disk drive andLANcard activity indicators) by covert channels; 
methods for dealing with extremely low-level optical emanations; exploitation 
of nonñline-of-sight, or diffuse, emanations; several interesting aspects of fiber 
optics, including dark fiber; and the opportunities afforded by stimulated emanations 
(Appendix A). 
ACM Transactions on Information and System Security, Vol. 5, No. 3, August 2002.
Information Leakage from Optical Emanations ≤ 279 
Fig. 8. Optical sum of ten random but correctly formatted data signals. (SIMULATION) 
8.1 Low-Level Optical Emanations 
While no evidence was found of Class III emanations from data storage devices, 
more investigation is needed to verify that disk and tape drive activity indicators 
experience no second-level effects due to (for instance) insufficient power 
supply regulation. And the wide variety and distribution of LAN cards suggests 
the possibility that at least one might show more than just Class II activity on 
the LEDs.3 
Other possible sources of compromising optical emanations include leakage 
from improperly terminated fiber optics or unconnected fiber optic ports. Alternative 
forms of attack are possible, including active attacks via optically 
emitting bugs operating outside of the visible spectrum that would be missed 
by conventional (RF) countersurveillance scanners. A passive collection system 
could operate over dark fiber; an accidental passive fiber optic tap would result 
if the end of a fiber strand were exposed to optical emanations from other 
devices in the room. 
8.2 The Possibility of NonñLine-of-Sight Interception 
Still unresolved is the question of whether diffuse emanations from multiple 
commingled or nonñline-of-sight sources can be profitably unraveled. Optical 
signals sum linearly; the optical sum of n linearly independent sources results 
in an extremely complicated signal (Figure 8). A room full of LED status indicators, 
even if individual sources are not directly observable, nevertheless can be 
seen to fill an entire area with a diffuse red glow. Light leakage around a door, 
3LAN cards on PCs are particularly interesting, given that the LEDs are on the back panel; when 
the computer is conventionally oriented on a desk by a window, the LEDs are clearly visible from 
outside. 
ACM Transactions on Information and System Security, Vol. 5, No. 3, August 2002.
280 ≤ J. Loughry and D. A. Umphress 
Table III. Decoding of Diffuse Emanations from State Transitions. SIMULATION 
tevent(πs) Transition Interpretation tnext∞event(πs) What can be deduced? 
104:16 " °1 start bit 208:33 
184:16 " °2 start bit 288:33 At least two signals exist. 
208:33 none °1 data0 D 1 312:50 
235:16 " °3 start bit 339:33 : : : three signals : : : 
248:16 " °4 start bit 352:33 : : : four signals : : : 
288:33 # °2 data0 D 0 392:50 
312:50 none °1 data1 D 1 416:66 
339:33 none °3 data0 D 1 443:50 
352:33 none °4 data0 D 1 456:50 
359:16 " °5 start bit 463:33 : : : five signals : : : 
362:16 " °6 start bit 466:33 : : : six signals : : : 
392:50 " °2 data1 D 1 496:66 
416:66 none °1 data2 D 1 520:83 
443:50 # °3 data1 D 0 547:66 
or a passive fiber optic tap,4 might provide an adversary with enough optical 
flux to begin to analyze it. 
8.2.1 Experiments with NonñLine-of-Sight Access. Sometimes things that 
are impossible in theory turn out to be feasible in practice. While it is true 
that in the general case of n random square wave signalsówhose amplitude, 
pulse width, and pulse repetition rate are unknownóthat a unique decomposition 
may not exist, in the real world, however, data signals are not random. 
EIA/TIA-232-E in particular is full of known values: start symbols, stop symbols, 
the number of data bits following a start symbol, and the guaranteed 
minimum and maximum duration of all symbols (the unit interval). Because 
the signals are not entirely random, but contain a small number of known 
values at certain fixed locations, it becomes possible to identify individual components 
(°1 through °n) with high probability. The left-hand side of Table III 
gives the timing and direction of state transitions during the first few hundred 
πs of the simulation shown in Figure 8. The algorithm works by scanning the 
received optical waveform from left to right until the first positive-going transition 
is found. The optical signal should be sampled at a rate at least three 
to five times the reciprocal of the smallest time difference between successive 
level transitions [McCarthy 2001]. Once a candidate transition is identified in 
the aggregate optical sum, any further activity on that particular component 
(°i) can be ruled out for at least one unit interval. Any transitions seen in the 
meantime must be the result of another, heretofore unknown signal (°iC1). By 
an iterative process of elimination, each individual signal in turn is teased out 
of the jumble. 
The unit interval is not difficult to guess from the Fourier spectrum of the 
times of transitions in the received signal (Figure 9). The peak in the curve, 
4A passive fiber optic tap might consist of as little as an unused strand of fiber, terminated at a 
patch panel inside the room but reserved for future use. Consequently, unused fiber optic ports 
should be capped when not in use. 
ACM Transactions on Information and System Security, Vol. 5, No. 3, August 2002.
Information Leakage from Optical Emanations ≤ 281 
Fig. 9. Fourier spectrum (real part) of the interval between transitions in the optical sum in Figure 
8. The peak in the curve, at approximately 104 πs, corresponds to the most likely unit interval. 
at 104 πs, corresponds to the most likely unit interval. In any case, the range 
of possible data formats is small enough simply to try all of the various possibilities 
until one of them yields intelligible data. Even an ambiguous solution 
might be of some value to an attacker, if the result were a data stream having 
some nonzero, but not catastrophic, bit error rate.5 As long as the individual 
signal components are not precisely aligned in timeówhich leads to ambiguous 
solutionsóthe analysis appears to be tractable. More work is clearly needed, on 
real signals, as a follow-on to the unrealistically low-noise simulation presented 
here. 
APPENDIXES 
A. STIMULATED EMANATIONS 
Not all sources of compromising optical emanations are naturally occurring. 
We describe two implementations of a Trojan horse that manipulates the 
LEDs on a standard keyboard to implement a high-bandwidth covert channel 
[Wray 1991]. This is an example of an active attack, mounted by an adversary 
against a device that is not normally vulnerable to compromising optical 
emanations. 
A.1 The Keyboard Considered as an Output Device 
Ever since the standardization of computer keyboards to the IBM layout, most 
computer keyboards have three LED indicators, for Caps Lock, Num Lock, and 
Scroll Lock, respectively. Interestingly, these LEDs are not directly connected 
to their associated keysóthe lights, in fact, are software controlled. 
5For example, if the most-significant bit of an 8-bit data word is not always 0, the data stream is 
not ASCII. Similarly, there are many disallowed values in EBCDIC that could be used to rule out 
this encoding as well. 
ACM Transactions on Information and System Security, Vol. 5, No. 3, August 2002.
282 ≤ J. Loughry and D. A. Umphress 
The PC keyboard is an intelligent device that communicates with the host 
computer over a bidirectional, synchronously clocked serial interface at approximately 
10 000 bits/s.6 
The capacity of the keyboard interface channel far exceeds the requirements 
of even the fastest typist. So long as the amount of data sent to the keyboard is 
limited, and does not interfere with processing of keystrokes, the excess bandwidth 
can be profitably employed by an attacker. 
A covert channel is a means of extracting data from a computer system 
in violation of the system security policy [Lampson 1973; National Computer 
Security Center 1993]. A high-bandwidth covert channel is considered to be one 
capable of transmitting data faster than 100 bits/s [Common Criteria Project 
Sponsoring Organizations 1999; United States Department of Defense 1985]. 
The covert channel described here has been demonstrated to work at speeds 
from 150 to 10 000 bits/s. 
A.2 Related Work 
The fact that keyboard LEDs can be manipulated has been known for a long 
time. Some operating systems provide the capability to control the keyboard 
indicators from a shell script; if not, then it is a relatively simple matter to 
program directly to the keyboard interface [van Gilluwe 1994]. 
A more recent paper describes another possible method for remotely monitoring 
the electrical signals inside a PC keyboard, together with some countermeasures. 
[Anderson and Kuhn 1999]. The only other published description 
of a covert channel employing keyboard LEDs appears in a work of fiction 
[Stephenson 1999], in which a character employs a similar technique to extract 
a small amount of critical information from his computer despite being under 
continuous surveillance. 
A.3 A Covert Channel in Software 
A successful covert channel running at up to 450 bits/swas demonstrated on the 
IBM PC/AT, several different Compaq ProLineas, and the Sun Microsystems 
SPARCstation 20 and Ultra 1 workstations. The attack was successful under 
MS-DOS, Microsoft Windows 3.1, Windows 95, and Windows 98, Windows NT 
3.5 and 4.0, and Sun Microsystems Solaris 2.5, 2.5.1, Solaris 7, and Trusted 
Solaris 2.5 and 2.5.1. A handful of machines could not be made to work, among 
them a Compaq LTE Lite 486/25E notebook. 
We found that activity on a single keyboard LED at 150 bits/s was not particularly 
noticeable during interactive use. Employing more than one LED at 
a time increases the probability of discovery but offers some compelling advantages. 
If all three LEDs are modulated identically, the optical output of the 
transmitter is tripled, greatly increasing the useful range. Alternatively, two 
or even three bits could be transmitted in parallel, increasing the bandwidth of 
the covert channel to approximately 450 bits/s. Experiments were run with (1) 
asynchronous parallel transmission using three LEDs, (2) synchronous serial 
6The exact speed was found to vary among different manufacturers. 
ACM Transactions on Information and System Security, Vol. 5, No. 3, August 2002.
Information Leakage from Optical Emanations ≤ 283 
Fig. 10. Modifications to the IBM PC/AT keyboard. 
transmission using single and biphase clocking, and (3) differential Manchester 
encoding. The latter yielded high reliability at the receiving end, but with all 
the activity on three LEDs, it was noticeable to the operator that something 
strange was going on. 
Appendix B contains example code implementing the covert channel under 
Solaris version 2.x.7 
A.4 Attacking the Hardware 
Even better results can be obtained through modification of the keyboard hardware. 
Depending on the details of a particular keyboard, the modifications may 
be as simple as moving a single wire. An IBM PC/AT keyboard was modified as 
an experiment. The Scroll Lock LED was cross-connected to the keyboard data 
signal, as shown in Figure 10. It was necessary to invert the keyboard data 
signal so that the LED would remain dark when the covert channel was idle. 
Fortunately, IBM provided a ready-made solution in the form of an unused gate 
on one of the chips. The optical output of the LED is now modulated directly 
by the 10 000 bits/s serial data stream in the keyboard cable. The Scroll Lock 
LED can be seen to flicker momentarily with keyboard activity, but the effect 
is not very noticeable. No software is required. 
Normal operation of the Scroll Lock LED is prevented, but the Scroll Lock 
function is not used very often. By a fortuitous coincidence, the normal behavior 
of the keyboard LEDs during the power-on self test (POST) function is unaffected; 
the functionality of the Scroll Lock key itself is also unchanged (except, 
of course, that the LED does not appear to work anymore.) 
Figure 11 shows the optical waveform obtained from a keyboard with the 
modifications of Figure 10. The upper trace of Figure 11 shows the intercepted 
optical signal; the lower two traces are the electrical signals on the keyboard 
data interface and the keyboard data clock. The bandwidth of the resulting 
7The authors demonstrated this technique privately in 1996. 
ACM Transactions on Information and System Security, Vol. 5, No. 3, August 2002.
284 ≤ J. Loughry and D. A. Umphress 
Fig. 11. Optical signal (top) obtained from a keyboard with the modifications of Figure 10. 
covert channel is greater than that of a software-only attack, but the information 
is in the form of keyboard scan codes, not ASCII. It requires a bit of 
translation on the receiving end, but also yields more information. Since accurate 
timing of both key-down and key-up events are reported, this method may 
provide enough information to compromise identity verification systems based 
on typing characteristics [Umphress and Williams 1985] or the generation of 
cryptographic keys [Garfinkel 1994]. 
A.4.1 Improving the Bandwidth of the Covert Channel. It is not difficult 
to imagine how a small investment in additional hardware could vastly improve 
the chances of a successful attack. An infrared (IR) LED chip could be 
co-encapsulated with a visible LED in the same package. If the two LEDs were 
connected back-to-back internally, only two leads would be required, and the 
Trojan LED would be indistinguishable from a standard component except under 
high magnification. Modification to the keyboard controller circuitry would 
be required to utilize the IR capability; as long as this is being done anyway, 
the following ìimprovementsî might be made to the controllerís firmware at the 
same time: 
óIncreasing the drive current to the IR emitter for correspondingly increased 
range 
óUse of more sophisticated channel encoding to reduce transmission errors 
and support higher speeds 
ACM Transactions on Information and System Security, Vol. 5, No. 3, August 2002.
Information Leakage from Optical Emanations ≤ 285 
óA timer and buffer memory to allow for a delay in sending until the keyboard 
has been idle for a while 
óEncryption and compression of the covert channel data 
óSender identification, to support multiple units in a single location 
óPattern matching capability, to look for specific information in the keyboard 
data stream 
óPreserving the normal functionality of the visible LED indicator. 
All but the first of these have been successfully demonstrated in software. 
Given that access to the hardware or surreptitious replacement would be necessary 
in order to emplace a hardware Trojan horse, concurrent implementation 
of all the above features would seem to pose little trouble. Modifications 
to firmware would be nearly undetectable, barring a close examination of the 
microcontroller object code. 
A.5 Conclusions 
This vulnerability potentially affects hundreds of millions of devices. It might 
be argued that keyboard LEDs lack sufficient brightness to be successfully exploited 
from a long distance. However, the authors once encountered a Compaq 
PC whose keyboard LEDs were bright enough to throw shadows on the ceiling. 
When tested, this keyboard was able to handle 450-bits/s communication on 
all three LEDs simultaneously without noticeably affecting response time. The 
software presented in Appendix B is small enough to be included in a computer 
virus, as described in Petitcolas et al. [1999]. 
A.6 Summary 
It has been shown that it is possible to cause the emission of compromising 
optical emanations in devices not normally vulnerable, by taking advantage 
of software-controlled LED indicators. The covert channel thereby created has 
a bandwidth of several hundred bits/second at least, and is compatible with 
standard techniques for exploiting compromising optical emanations described 
in the previous paper. 
B. SENDING DATA THROUGH THE KEYBOARD 
The following C program implements the software version of the covert channel 
under Solaris version 2.x. It transmits ASCII data by modulating the Caps 
Lock LED with serial data at 50 bits/s. A similar program, written in Intel x86 
assembly language and incorporating additional functionality,8 required less 
than 1500 bytes of memory. 
8The program installed itself as an interrupt handler and hooked the keyboard interrupt. It copied 
all keyboard activity while waiting for the keyboard to become idle. After four hours of no keyboard 
activity, the contents of the buffer were transmitted. If any keyboard activity was detected while the 
program was busy transmitting, it immediately stopped sending, restored the state of the keyboard 
LEDs, and resumed waiting. 
ACM Transactions on Information and System Security, Vol. 5, No. 3, August 2002.
286 ≤ J. Loughry and D. A. Umphress 
/* 
// sl.c -- a covert channel using the Caps Lock LED. 
// 
// For Solaris 2.x on SPARC; compile with ${CC} sl.c -lposix4 
*/ 
#include <fcntl.h> 
#include <stdio.h> 
#include <stdlib.h> 
#include <sys/kbio.h> 
#include <sys/kbd.h> 
#include <time.h> 
#include <unistd.h> 
#define SPEED 50 /* data transmission speed (bits per second) */ 
void set_led (int fd, char *data); 
void time_led (int fd, char *data); 
void perror_exit (char *function_name); 
/* set up a 20 millisecond intersymbol delay */ 
struct timespec min, max = { 0, 1000000000 / SPEED }; 
int 
main (void) 
{ 
char message[] = "My credit card number is 1234 5678 910 1112."; 
char restore_data; 
char *p = &message[0]; 
int fd; 
/* open the keyboard device */ 
if ((fd = open ("/dev/kbd", O_RDONLY)) < 0) 
perror_exit ("open"); 
/* save the state of the keyboard LEDs */ 
if (ioctl (fd, KIOCGLED, &restore_data) < 0) 
perror_exit ("ioctl"); 
while (*p) { 
char data = LED_CAPS_LOCK; 
int i; 
/* start bit is a "1" */ 
time_led (fd, &data); 
ACM Transactions on Information and System Security, Vol. 5, No. 3, August 2002.
Information Leakage from Optical Emanations ≤ 287 
/* send 8 bits, least significant first */ 
for (i = 0; i < 8; i++) { 
data = *p >> i & 1 ? LED_CAPS_LOCK : 0; 
time_led (fd, &data); 
}
/* stop bit is a "0" */ 
data = 0; 
time_led (fd, &data); 
/* next character of message */ 
p++; 
}
/* restore state of the keyboard LEDs */ 
set_led (fd, &restore_data); 
return (close (fd)); 
}
/* turn keyboard LEDs on or off */ 
void 
set_led (int fd, char *data) 
{ 
if (ioctl (fd, KIOCSLED, data) < 0) 
perror_exit ("ioctl"); 
}
/* transmit one bit */ 
void 
time_led (int fd, char *data) 
{ 
set_led (fd, data); 
nanosleep (&min, &max); 
}
/* display an error message and quit */ 
void 
perror_exit (char *function_name) 
{ 
perror (function_name); 
exit (1); 
} 
ACM Transactions on Information and System Security, Vol. 5, No. 3, August 2002.
288 ≤ J. Loughry and D. A. Umphress 
ACKNOWLEDGMENTS 
The authors wish to thank the anonymous reviewers; their careful reading and 
insightful comments helped catch a number of errors that otherwise would have 
crept into publication. Annie Cruz of Washington Mutual Bank and Eduard 
Telders of PEMCO Financial Services also provided valuable assistance and 
encouragement with the preparation of this article. 
REFERENCES 
AGILENT TECHNOLOGIES. 1999. T-1 34 
(5 mm) Diffused LED Lamps Technical Data. Agilent Technologies. 
Data sheet 5968-4161E (2/99). 
ANDERSON, R. J. 2001. Security Engineering: A Guide to Building Dependable Distributed Systems. 
Wiley, New York. 
ANDERSON, R. J. AND KUHN, M. G. 1999. Soft tempestóan opportunity for NATO. In Protecting 
NATO Information Systems in the 21st Century. NATO Research & Technology Organisation, 
Washington, D.C. 
BLACK, U. 1996. Physical Layer Interfaces and Protocols, 2nd ed. IEEE Computer Society Press, 
Los Alamitos, Calif. 
COMMON CRITERIA PROJECT SPONSORING ORGANIZATIONS. 1999. Common Criteria for Information 
Technology Security Evaluation. Common Criteria Project Sponsoring Organizations. CCIMB- 
99-031, Version 2.1. 
ELECTRONIC INDUSTRIES ASSOCIATION, ENGINEERING DEPARTMENT. 1991. Interface Between Data Terminal 
Equipment and Data Circuit-Terminating Equipment Employing Serial Binary Data Interchange. 
Electronic Industries Association, Engineering Department. EIA/TIA-232-E. 
EXFO ELECTRO-OPTICAL ENGINEERING, INC. 1999. LFD-100 Live Fiber Detector. EXFO Electro- 
Optical Engineering, Inc. Data sheet SPLFD100.4AN. 
FITES, P. AND KRATZ, M. P. 1993. Information Systems Security: A Practitionerís Reference. Van 
Nostrand Reinhold, New York. 
GAGLIARDI, R. 1995. Optical Communications, 2nd ed. Wiley, New York. 
GARFINKEL, S. 1994. PGP: Pretty Good Privacy. OíReilly & Associates, Sebastopol, California. 
HEWLETTñPACKARD COMPANY. 1993a. HFBR-4663 Single Chip 10BASEñFL Transceiver Technical 
Data. HewlettñPackard Company. Data sheet 5091-7391E. 
HEWLETTñPACKARD COMPANY. 1993b. Low-Cost Fiber-Optic Links for Digital Applications up to 155 
MBd. HewlettñPackard Company. Application Bulletin 78, 5091-9102E. 
HODARA, H. 1991. Secure fiberoptic communications. In Proceedings of Symposium on Electromagnetic 
Security for Information Protection. Fondazione Ugo Bordoni, Rome, Italy. 
JOHNSON, P. 1995. Circuit adapts signals for visual perception. Electronic Design News 40, 21 (12 
October), 104. 
KUHN, M. G. 2002. Optical time-domain eavesdropping risks of CRT displays. In Proceedings of 
the 2002 IEEE Symposium on Security and Privacy. IEEE Computer Society, Oakland, California. 
KUHN, M. G. AND ANDERSON, R. J. 1998. Soft tempest: Hidden data transmission using electromagnetic 
emanations. In Proceedings of Information Hiding, Second International Workshop, 
D. Aucsmith, ed. SpringerñVerlag, Portland, Oregon, 15ñ17. 
LAMPSON, B.W. 1973. A note on the confinement problem. Commun. ACM 16, 10 (Oct.), 613ñ615. 
LANCASTER, D. 1980. TTL Cookbook. Howard W. Sams, Indianapolis, Ind. 
LATHROP, D. L. 1992. Security aspects of wireless local area networks. Comput. Secur. 11, 421ñ 
426. 
MCCARTHY, D. C. 2001. Faster vs. denser: Networks reach another crossroad. Photon. Spectra 35, 
9 (Sept.), 110ñ118. 
MORRIS, J. 1996. Re: blinking lights on computers. Article h55ni3a$bm3top.mitre.orgi, in 
USENET newsgroup alt.folklore.computers. 
NATIONAL COMPUTER SECURITY CENTER. 1988. Glossary of Computer Security Terms. National Computer 
Security Center. NCSC-TG-004, Version 1. 
NATIONAL COMPUTER SECURITY CENTER. 1993. A Guide to Understanding Covert Channel Analysis 
of Trusted Systems. National Computer Security Center. NCSC-TG-030, Version 1. 
ACM Transactions on Information and System Security, Vol. 5, No. 3, August 2002.
Information Leakage from Optical Emanations ≤ 289 
NATIONAL SECURITY AGENCY. 1992. NACSIM 5000 TEMPEST Fundamentals. National Security 
Agency, Fort George G. Meade, Md. http://cryptome.org/nacsim-5000.htm. 
NATIONAL SECURITY AGENCY. 1994. Specification NSA No. 94-106, Specification for Shielded Enclosures. 
National Security Agency, Fort George G. Meade, Md. http://cryptome.org/nsa-94-104. 
htm. 
NATIONAL SECURITY AGENCY. 1995. TEMPEST/2-95 Red/Black Installation Guidance. National 
Security Agency, Fort George G. Meade, Md. http://cryptome.org/tempest-2-95.htm. 
PARADYNE CORPORATION. 1985. InfoLock Model 2811-11 Installation and Operation Manual, 1st 
ed. Paradyne Corporation. 2811-A2-GN32-00. 
PERRY, T. S. 1995. M. George Craford. IEEE Spect. 32. 2 (February), 52ñ55. 
PETITCOLAS, F. A., ANDERSON, R. J., AND KUHN, M. G. 1999. Information hidingóA survey. Proc. 
IEEE 87, 7 (July), 1062ñ1078. 
PROAKIS, J. G. AND SALEHI, M. 1994. Communication Systems Engineering. Prentice-Hall, 
Englewood Cliffs, N.J. 
PROCTOR, N. E. AND NEUMANN, P. G. 1992. Architectural implications of covert channels. In Proceedings 
of the 15th National Computer Security Conference. National Institute of Standards and 
Technology, National Computer Security Center, Baltimore, Md., 28ñ43. 
RUSSELL, D. AND GANGEMI, G. 1991. Computer Security Basics. OíReilly & Associates, Sebastopol, 
Calif. 
SMULDERS, P. 1990. The threat of information theft by reception of electromagnetic radiation from 
RS-232 cables. Comput. Secur. 9, 1, 53ñ58. 
STEPHENSON, N. 1999. Cryptonomicon. Avon Books, New York. 
TELECOMMUNICATIONS INDUSTRY ASSOCIATION. 1996. Standard for StartñStop Signal Quality 
for Non-Synchronous Data Terminal Equipment. Telecommunications Industry Association. 
TIA/EIA-404-B. 
UMPHRESS, D. AND WILLIAMS, G. 1985. Identity verification through keyboard characteristics. Int. 
J. ManñMachine Studies 23, 263ñ273. 
UNITED STATES DEPARTMENT OF DEFENSE. 1985. Trusted Computer System Evaluation Criteria. 
United States Department of Defense. DOD 5200.28-STD. 
UNITED STATES DEPARTMENT OF DEFENSE. 1987. Red/Black EngineeringñInstallation Guidelines. 
United States Department of Defense. MIL-HDBK-232A. 
VAN ECK, W. 1985. Electromagnetic radiation from video display units: An eavesdropping risk? 
Comput. Secur. 4, 269ñ286. 
VAN GILLUWE, F. 1994. The Undocumented PC. AddisonñWesley Publishing Company, Reading, 
Mass. 
WILKINS, J. 1641. Mercury, or the Secret and Swift Messenger. I. Norton, London. 
WRAY, J. C. 1991. An analysis of covert timing channels. In Proceedings of the 1991 IEEE Computer 
Society Symposium on Research in Security and Privacy (Oakland, Calf.). IEEE Computer 
Society, Los Alamitos, Calif. 2ñ7. 
WRIGHT, P. 1987. Spycatcher: The Candid Autobiography of a Senior Intelligence Officer. Viking 
Press, New York. 
Received April 2001; revised February 2002 and March 2002; accepted March 2002 
ACM Transactions on Information and System Security, Vol. 5, No. 3, August 2002.
